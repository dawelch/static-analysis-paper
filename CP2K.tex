CP2K \cite{hutter2014cp2k} is a popular molecular simulation package that includes numerous routines for quantum chemistry and solid-state physics, including ab initio molecular dynamics (AIMD) \cite{marx2009ab}. Molecular dynamics within the Born-Oppenheimer (BO) regime simulates atomic motion  according to classical mechanics, while forces may be calculated using any type of physics. For AIMD, the forces are calculated using quantum mechanics, via derivatives of the system's electronic potential energy surface. This energy is usually derived with a self-consistent field (SCF) calculation with density functional theory (DFT) \cite{vandevondele2012linear,hutter2014cp2k}, which involves an iterative solver that must converge before the energy is known, and tensor contractions of large tensors representing electron orbitals. Therefore, each time-step requires considerably more computational time and memory than classical molecular dynamics. However, AIMD is able to explore essential chemical properties such as time-dependent changes in polarization, charge densities, protonation states, and bond breaking and formation, all of which are inaccessible to classical molecular dynamics and of great interest to physicists, chemists and molecular biologists.

Until recently, AIMD simulations on systems with thousands of atoms for timescales longer than a few picoseconds was not possible. Currently, several parallel implementations of the DFT-based SCF calculation have exploited massive parallelization on HPC systems to obtain the SCF-based electronic energy of such large systems in under a minute \cite{vasp_bench,kresse1996efficient,cp2k_bench,vandevondele2012linear}. This opens up the possibility of simulating, for the first time, electronic-structure-dependent properties as time series extending to hundreds of picoseconds of simulation time, which can be compared to experimental results \cite{gillan2016perspective, pestana2017ab, hassanali2013proton, milovanovic2018new, sellner2013charge}.

The linear-scaling SCF (LS-SCF) calculation \cite{vandevondele2012linear} in CP2K has made use of an algorithm that exploits physical locality to decompose the SCF calculation into what effectively becomes sparse matrix multiplication, which is then approached with their own auto-tuning small matrix multiplication library, \texttt{libsmm} \cite{borvstnik2014sparse}. An Intel-optimized version of this library has also been created by programmers working for INtel \cite{heinecke2016libxsmm}. In addition, a GPU-based version is available, and the procedures are all incorporated into the Distributed Block Compressed Sparse Row (DBCSR) library \cite{borvstnik2014sparse,schutt2016gpu}. Despite these efforts and other extensive optimizations including  threading, use of parallel libraries such as ScaLAPACK and threaded FFTW, and the ability to scale to over a thousand nodes CP2K cannot scale at capability-level on leadership 

While other simulation programs have been able to scale to thousands of nodes on HPC systems, AIMD programs, 