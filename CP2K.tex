CP2K \cite{hutter2014cp2k} is a popular molecular simulation package, written in Fortran 95, that includes numerous routines for quantum chemistry and solid-state physics, including \ac{AIMD} \cite{marx2009ab}.
Molecular dynamics within the \ac{BO} regime simulates atomic motion according to classical mechanics, while forces may be calculated using any type of physics.
For \ac{AIMD}, the forces are calculated using quantum mechanics, via derivatives of the system's electronic potential energy surface.
This energy is usually obtained with a \ac{SCF} calculation with \ac{DFT} \cite{vandevondele2012linear,hutter2014cp2k}, which involves an iterative solver that must converge before the energy is known, and tensor contractions of large tensors representing electron orbitals.
Therefore, each time-step requires considerably more computational time and memory than classical molecular dynamics.
However, \ac{AIMD} is able to explore essential chemical properties such as time-dependent changes in polarization, charge densities, protonation states, and bond breaking and formation, all of which are inaccessible to classical molecular dynamics and of great interest to physicists, chemists and molecular biologists.

Until recently, \ac{AIMD} simulations on systems with thousands of atoms for timescales longer than a few picoseconds was not possible.
Currently, several parallel implementations of the \ac{DFT} based \ac{SCF} calculation have exploited massive parallelization on \acs{HPC} systems to obtain the \ac{SCF} based electronic energy of such large systems in under a minute \cite{vasp_bench,kresse1996efficient,cp2k_bench,vandevondele2012linear}.
This opens up the possibility of simulating, for the first time, electronic-structure-dependent properties as time series extending to hundreds of picoseconds of simulation time, which can be compared to experimental results \cite{gillan2016perspective, pestana2017ab, hassanali2013proton, milovanovic2018new, sellner2013charge}.

The \ac{LS-SCF} calculation \cite{vandevondele2012linear} in CP2K has made use of an algorithm that exploits physical locality to decompose the \ac{SCF} calculation into what effectively becomes sparse matrix multiplication, which is then approached with their own auto-tuning small matrix multiplication library, \texttt{libsmm} \cite{borvstnik2014sparse}.
An Intel-optimized version of this library, \texttt{libxsmm}, has also been created for CP2K by programmers working for Intel \cite{heinecke2016libxsmm}.
In addition, a \acs{GPU} based version is available, and the procedures are all incorporated into the \ac{DBCSR} library \cite{borvstnik2014sparse,schutt2016gpu}, which is the engine for the \ac{LS-SCF} calculation.
Despite these efforts and other extensive optimizations including excellent \acs{MPI} based decomposition, OpenMP threading, and use of parallel libraries such as ScaLAPACK and threaded \ac{FFTW}, CP2K cannot scale to much more than a thousand nodes on supercomputers, resulting in a lack of ability to utilize over 5\% of systems such as the \acs{OLCF}'s Titan, while capability level programs often are able to use much more.
It is programs such as CP2K, therefore, which provide state of the art tools for obtaining cutting edge scientific results, that can benefit greatly from performance analysis tools beyond profilers, in order to potentially further increase their ability to make efficient use of leadership computing resources.
