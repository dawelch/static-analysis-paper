%TODO: benefit factors still need to be explained
Future exascale environments need intelligent data management across the heterogeneous memory components, so as to maximise application performance.
For that reason, it is necessary to be able to understand the impact of accessing each data object to the application runtime.
This knowledge enables efficient data placement and migration decisions.
To that extent, CoMerge \cite{Doudali:2017:CTE:3132402.3132418} is a memory sharing schema that maximises the hardware utilisation and performance across collocated applications, via prioritising allocations of data objects whose access rate and pattern greatly impacts the runtime.
 
CoMerge assumes a hybrid system with two memory components, \textit{FastMem} and \textit{SlowMem}, where the second one is configured to have 0.2x the bandwidth and 5x the latency of the first one, which is a representative difference of DRAM and \ac{NVM} technologies, respectively.
Thus, in such a system, where dataset sizes span allocations across the different memory components, there will be an application performance slowdown from the ideal case, where all data could be serviced from FastMem.
For this reason, it is important to capture the extent of slowdown reduction that is facilitated by the placement and access of each data object to FastMem.
The extent of slowdown reduction directly correlates to the access rate, density and pattern for each data object.
We capture this by observing the overall application runtime, for single object allocations to FastMem while the rest of the dataset is serviced from SlowMem.
Experimental analysis shows that there are many applications, where only few objects contribute significantly to the slowdown minimisation.
We refer to the degree to which having any such object in FastMem brings us to the previously described ideal case as its benefit factor.
That manual tiering methodology leads to the inference of a data object ordering, that corresponds to a priority order for FastMem allocations prior to application execution.

Given this priority order, CoMerge proposes a memory sharing schema, where data object orderings of collocated applications are merged into a global order.
Then objects are allocated in FastMem until full capacity, following that global order, prior to the actual application execution.
This schema is able to maximize the utilisation of the memory hardware, compared to static partitioning divisions.
More importantly, CoMerge ensures that the critical to performance data objects across all applications are prioritised for FastMem allocations, leading to performance improvements across all collocated applications.
Therefore, CoMerge highlights a representative use case, where the utilisation of data object level information enables clever placement decisions in a hybrid memory system.

However, this manual exploration of data object level impact to performance is infeasible for applications with large numbers of objects and without the expertise and effort of programmers that have contributed to the application development.
This is the reason why in this work we aim to obviate this process through compiler assisted techniques.
