\documentclass[letterpaper, 10 pt, conference]{ieeeconf}

\IEEEoverridecommandlockouts
%\overrideIEEEmargins

%\usepackage{subcaption}
%\captionsetup{compatibility=false}
\usepackage{authblk}
\usepackage{amsmath}
\usepackage{graphicx}
% \renewcommand\UrlFont{\color{blue}\rmfamily}
\usepackage{color}
\usepackage{listings}

\definecolor{gray}{rgb}{0.5,0.5,0.5}
\definecolor{mauve}{rgb}{0.58,0,0.82}

\lstset{ 
  language=SQL,                   % the language of the code
  basicstyle=\footnotesize,       % the size of the fonts that are used for the code
  numbers=left,                   % where to put the line-numbers
  numberstyle=\tiny\color{gray},  % the style that is used for the line-numbers
  stepnumber=1,                   % the step between two line-numbers. If it's 1, each line 
                                    % will be numbered 
  backgroundcolor=\color{white},  % choose the background color. You must add \usepackage{color}
  showspaces=false,               % show spaces adding particular underscores
  showstringspaces=false,         % underline spaces within strings
  showtabs=false,                 % show tabs within strings adding particular underscores
  rulecolor=\color{black},        % if not set, the frame-color may be changed on line-breaks within not-black text (e.g. commens (green here))
  tabsize=1,                      % sets default tabsize to 2 spaces
  captionpos=b,                   % sets the caption-position to bottom
  breaklines=true,                % sets automatic line breaking
  breakatwhitespace=true,        % sets if automatic breaks should only happen at whitespace
  title=\lstname,                 % show the filename of files included with \lstinputlisting;
                                  % also try caption instead of title
  keywordstyle=\color{blue},          % keyword style
  commentstyle=\color{dkgreen},       % comment style
  stringstyle=\color{mauve},         % string literal style
  escapeinside={\%*}{*)},            % if you want to add LaTeX within your code
  xleftmargin=15pt,
  morekeywords={*,...}              % if you want to add more keywords to the set
}

\usepackage[nolist]{acronym}
%\ProvideAcroEnding{possessive}{'s}{'s}
%\ExplSyntaxOn
%\NewAcroCommand \acg{
%	\acro_possessive:
%	\acro_use:n{#1}
%}
%\ExplSyntaxOff
\begin{acronym}
\acro{AIMD}{ab initio molecular dynamics}
\acro{AMO}{atomic memory operation}
\acro{AMR}{adaptive mesh refinement}
\acro{API}{Application Programming Interface}
\acrodefplural{API}[APIs]{Application Programming Interfaces}
\acro{AST}{abstract syntax tree}
\acro{BO}{Born-Oppenheimer}
\acro{CAF}{Coarray Fortran}
\acro{CAS}{compare and swap}
\acro{CPU}{central processing unit}
\acrodefplural{CPU}[CPUs]{central processing units}
\acrodefplural{CRDP}[CRDP]{Computational Research and Development Programs}
\acro{CSV}{comma-separated values}
\acro{DBCSR}{distributed block compressed sparse row}
\acro{DFT}{density-functional theory}
\acro{DoD}{Department of Defense}
\acro{DoE}{Department of Energy}
\acro{ESSC}{Extreme Scale Systems Center}
\acro{FDO}{feedback directed optimisation}
\acrodefplural{FDO}[FDOs]{feedback directed optimisations}
\acro{FFTW}{fastest fourier transform in the west}
\acro{GCC}{GNU Compiler Collection}
\acro{GPU}{graphics processing unit}
\acrodefplural{GPU}[GPUs]{graphics processing units}
\acro{HBM}{high bandwidth memory}
\acro{HCA}{host channel adapter}
\acro{HPC}{high performance computing}
\acro{IR}{intermediate representation}
\acro{I/O}{input/output}
\acro{JIT}{just in time}
\acro{LS-SCF}{linear-scaling self-consistent field}
\acro{MPI}{Message Passing Interface}
\acro{NUMA}{non-uniform memory access}
\acro{NVM}{non-volatile memory}
\acro{OFA}{Open Fabrics Alliance}
\acro{OLCF}{Oak Ridge Leadership Computing Facility}
\acro{ORNL}{Oak Ridge National Laboratory}
\acro{PE}{processing element}
\acro{PGAS}{Partitioned Global Address Space}
\acro{RDMA}{remote direct memory access}
\acro{RMA}{remote memory access}
\acro{SCF}{self-consistent field}
\acro{SCoP}{static control part}
\acrodefplural{SCoP}[SCoPs]{static control parts}
\acro{SHOC}{Scalable Heterogeneous Computing}
\acro{SIMD}{single instruction, multiple data}
\acro{SPMD}{single program, multiple data}
\acro{SQL}{Structured Query Language}
\acro{UCP}{UC-Protocols}
\acro{UCS}{UC-Services}
\acro{UCT}{UC-Transports}
\acro{UCX}{Unified Communication X}
\acro{XML}{Extensible Markup Language}
\acro{XSLT}{Extensible Stylesheet Language Transformations}
\end{acronym}


\begin{document}
%\title{A Database for program analysis}
\title{\LARGE \bf
A Program Analysis Tool to Query Static and Dynamic Characteristics of HPC Applications
}
%\author[1, 2]{Aaron Welch}
%\author[1]{Ada Sedova}
%\author[1]{Oscar Hernandez}
%\author[1]{Terry Jones}
%\author[3]{Thaleia Dimitra Doudali}
%\author[3]{Ada Gavrilovska}
%\author[2]{Barbara Chapman}
%\affil[1]{Oak Ridge National Laboratory, Oak Ridge, TN}
%\affil[2]{University of Houston, Houston, TX}

\maketitle
\begin{abstract}
Emerging HPC platforms are becoming extraordinarily difficult to program as a result of complex, deep and heterogeneous memory hierarchies, heterogeneous cores, and the need to divide work and data among them.
New programming models and libraries are being developed to aid in porting efforts of application developers, but substantial code restructuring is still necessary to fully make use of these new technologies.
To do this effectively, these developers need information about their source code characteristics, including static and dynamic (e.g. performance) information to direct their optimization efforts and make key decisions.
On the other hand, system administrators need to understand how users are using the software stack and resources on their systems to understand what software they need to provide to improve the productivity of the users on a platform.

In this paper we describe a program database that connects compiler and profiler information that can be used to query program characteristics within an application or across multiple applications on a given platform.
Static and dynamic data about applications are collected and stored together in an \acs{SQL} database that can be queried by either a developer or a system administrator.
We will demonstrate the capabilities of this tool via a real-world example from application-driven case studies that aims at understanding the use of scientific libraries on a routine from the molecular simulation application CP2K.

%\keywords{static analysis \and dynamic analysis \and databases.}
\end{abstract}
\section{Introduction}
\label{sec:intro}
As we move toward exascale systems, memory hierarchies are becoming increasingly complex and heterogenous and they are expected to become even more so in the future.
%TODO: "compute in memory" 
%TODO: "How much of these memories are exposed via the programming model? Do we want that? or do we want to keep the programming models "intact" and just add implicit hints to control the placement of variables to different types of memories. If so, would a tool like this one help? This will definately provide less restructuring. 
Though we know that such new memory technologies like \ac{HBM} and \ac{NVM} require novel approaches to application design, what is less clear is how to address the new architectures featuring them to optimally make use of these technologies or how to efficiently implement such strategies portably using programming models.
Furthermore, considering the potentially costly process of porting applications we need to explore semi or full automated approaches to optimize existing codes written using defacto programming models.

%TODO: insert references here
There are a number of efforts toward extending libraries, and programming models that reduce the burden on the application developers with respect to managing these new architectures, but we will not be covering that angle. 
Instead, we will be focusing on  minimally intrusive, low overhead methods (implemented via tools) for identifying memory intensive regions of code so that developers may make better decisions regarding what parts of the application to focus their efforts on.
Typical performance analysis of applications via tools often focuses on computational, network, and I/O resources, however this is no longer adequate for determining how to achieve the greatest use of these new memory hierarchies.
%I think this sentence needs to say analysis to place the data to different memories -- place data structure X to memory Y. You need to mention this in this sentence.
Instead, the focus will increasingly need to shift toward analysing applications from the point of how data structures are being used or accessed, and to this end we look to methods of determining memory hotspots in algorithms and data structures.

%Can you put what compiler analysis: examples are loopnest analysis - polyhedral, data structure accesses (what data you are accessing in loops and how), variable information (types of variables such as arrays, # dimensions, etc), static analysis on the shapes of data structures and iteration spaces.  
We have created a way of performing static analysis of code by exporting information directly from the compiler into a relational database that we may later query.
%Dyanmic information in reality is sample based profiling, where with samples we can determine which statements are more time consuming on a program. 
We then combine this database with information obtained through traditional dynamic analysis tools in order to find performance critical allocations in an automated fashion.
The goal is to determine with as great an accuracy as possible which data structures influence application performance the most so that the developer may make better decisions with respect to data placement.
As we have previously explored this problem in a manual fashion (see section~\ref{sec:prevwork})\cite{Doudali:2017:CTE:3132402.3132418}, this work will focus on recreating the results we found by hand with the analysis tools described in this paper.
Furthermore, this work is currently focused on the analysis of Fortran code for simplicity, though it can be applied in much the same way to C/C++.

The rest of this paper is organised as follows: Section~\ref{sec:background} provides some background on the tools/methods used along with related work, Section~\ref{sec:prevwork} briefly details the relevant parts of our past work that we are attempting to replicate with our automated analysis, Section~\ref{sec:analysis} describes in detail the actual analysis performed, and Section~\ref{sec:results} shows the results of the analysis and how they compare to our prior manual testing.%, Section~\ref{sec:conclusion} evaluates the analysis and its results with respect to its usefulness on a larger scale, and Section~\ref{sec:future} outlines the next steps for improving this work further.

\section{Motivation}
\label{sec:motivation}
According to the center of accelerated application readiness at ORNL, 80\% of application porting efforts to OLCF Titan 
%(cite Titan page) 
(a \acs{GPU}-based system) was spent understanding and restructuring the code and 20\% on adding a new on-node parallel programming model (e.g. CUDA, OpenMP/OpenACC, etc). 
%(Citation needed for the 20 percent thing)
Worse yet is that much of this effort often does not translate well from one application to another, making it a tedious and costly affair.
We expect this aspect of porting to be equally if not even more burdensome for other and future systems.
Ideally, what a user wants is to be able to use a portable programming model that will allow it to build the source code as it is, swap library implementations in and out across systems, and run the code with different runtime parameters (e.g. \# of threads, MPI ranks, etc) in order to port code to multiple platforms more efficiently. This would allow us to relieve some of the strain placed on application developers in porting their code to future system architectures.
It also helps in avoiding the need to maintain multiple versions or code paths to support all target architectures, which is another problem that still plagues many applications.

Another issue many applications face is that even after initial porting efforts to use new software and hardware technologies, they may have difficulty with scaling to exploit the full extent of the system and may not use the resources efficiently and obtain the expected performance characteristics.
It is rarely easy to discover and understand the sources of inefficiencies due both to code and target architecture complexity as well as the scale.
As such, being able to quickly and easily investigate large code bases as well as identify efficient / inefficient code patterns is key to application developers being able to port more effectively.

%TODO: what are other models/libraries/tools doing to address this?

%TODO: OpenMP memory features? (oscar?)

%For example, a CP2K application developer knows that scalapack, a linear algebra library, is linked in his code but doesn't understand how it is being used in its code and how it is affecting the performance of the code on a given system.
%Ideally, she is interested to know how Scalapack is affecting the performance of a particular solver or calculation and how the parallel communication is affecting the performance on a given system.  

%She noticed that on a particular system Eos (a Xeon Phi based system) her code outperforms a GPU-based system Titan but doesn't know if its related to the implementation of the library, or how it is used in the code.
%She wants to understand why this is not an easy task as she is observing a lot of  performance difference.
%She thinks it may be related to the way the library is being invoked in the multi-threaded region of the code or it may be related on the quality of implementation of the library on a given system that favors one architecture over the other. 

%Furthermore, she wants to communicate her scientific library requirements to the system to the system administrator to make sure future systems have the proper library support for her code. 
 


%One last thing that I wouldn't mind knowing about, if it's possible, is how the code handles openMP vs. MPI use. My benchmarks of this calculation on Eos and Titan showed that using many MPI ranks per node, even 32 on Eos, gave better performance than any use of openMP at all! I thought this was interesting. It's possible that the MPI implementation is really well done and the openMP is not. Or, it could just be something about the calculation itself. I think this is actually a very interesting CS problem.

%Thanks!

%-A

\subsection{CP2K}
\label{sec:cp2k}
CP2K \cite{hutter2014cp2k} is a popular molecular simulation package, written in Fortran 95, that includes numerous routines for quantum chemistry and solid-state physics, including \ac{AIMD} \cite{marx2009ab}.
Molecular dynamics within the \ac{BO} regime simulates atomic motion according to classical mechanics, while forces may be calculated using any type of physics.
For \ac{AIMD}, the forces are calculated using quantum mechanics, via derivatives of the system's electronic potential energy surface.
This energy is usually obtained with a \ac{SCF} calculation with \ac{DFT} \cite{vandevondele2012linear,hutter2014cp2k}, which involves an iterative solver that must converge before the energy is known, and tensor contractions of large tensors representing electron orbitals.
Therefore, each time-step requires considerably more computational time and memory than classical molecular dynamics.
However, \ac{AIMD} is able to explore essential chemical properties such as time-dependent changes in polarization, charge densities, protonation states, and bond breaking and formation, all of which are inaccessible to classical molecular dynamics and of great interest to physicists, chemists and molecular biologists.

Until recently, \ac{AIMD} simulations on systems with thousands of atoms for timescales longer than a few picoseconds was not possible.
Currently, several parallel implementations of the \ac{DFT} based \ac{SCF} calculation have exploited massive parallelization on \acs{HPC} systems to obtain the \ac{SCF} based electronic energy of such large systems in under a minute \cite{vasp_bench,kresse1996efficient,cp2k_bench,vandevondele2012linear}.
This opens up the possibility of simulating, for the first time, electronic-structure-dependent properties as time series extending to hundreds of picoseconds of simulation time, which can be compared to experimental results \cite{gillan2016perspective, pestana2017ab, hassanali2013proton, milovanovic2018new, sellner2013charge}.

The \ac{LS-SCF} calculation \cite{vandevondele2012linear} in CP2K has made use of an algorithm that exploits physical locality to decompose the \ac{SCF} calculation into what effectively becomes sparse matrix multiplication, which is then approached with their own auto-tuning small matrix multiplication library, \texttt{libsmm} \cite{borvstnik2014sparse}.
An Intel-optimized version of this library, \texttt{libxsmm}, has also been created for CP2K by programmers working for Intel \cite{heinecke2016libxsmm}.
In addition, a \acs{GPU} based version is available, and the procedures are all incorporated into the \ac{DBCSR} library \cite{borvstnik2014sparse,schutt2016gpu}, which is the engine for the \ac{LS-SCF} calculation.
Despite these efforts and other extensive optimizations including excellent \acs{MPI} based decomposition, OpenMP threading, and use of parallel libraries such as ScaLAPACK and threaded \ac{FFTW}, CP2K cannot scale to much more than a thousand nodes on supercomputers, resulting in a lack of ability to utilize over 5\% of systems such as the \acs{OLCF}'s Titan, while capability level programs often are able to use much more.
It is programs such as CP2K, therefore, which provide state of the art tools for obtaining cutting edge scientific results, that can benefit greatly from performance analysis tools beyond profilers, in order to potentially further increase their ability to make efficient use of leadership computing resources.

\section{Related Work}
\label{sec:related}
Research on static analysis of code has a long history, and has both taken many forms and addressed numerous problem domains\cite{Andrade:2012:SAW:2355585.2355593}\cite{1194988}.
One key challenge is to determine how variables are used and data travels through them in some form of dataflow analysis\cite{Feautrier1991}.
This is also commonly used to discover more complicated patterns of memory access, for enabling optimisations such as loop tiling to make better use of cache performance.
Among the more advanced methods compilers use to analyse code is to employ the polyhedral model\cite{Cousot:1978:ADL:512760.512770}\cite{Bagnara:2009:APC:1628316.1628385}\cite{benabderrahmane.10.cc}.
The polyhedral method is ideally suited for representing and reasoning about loops, although is generally restricted to operating on affine loop nests. 
In addition, high relative computational expense has traditionally limited its practical use in compilation \cite{DBLP:journals/entcs/Simon10a}.
Nonetheless, a lot of work has been done on it for more precisely determining dependences \cite{Vasilache:2006:VDA:1183401.1183448} as well as for more advanced optimisation techniques\cite{Nieuwenhuizen2014AutovectorizationUP}\cite{5260526}.

As well researched as static analysis methods are, they cannot provide a comprehensive analysis of applications on their own.
Static analysis is limited to providing functional data about code (e.g. routine invocations, control/data flow, etc), but can't provide non-functional data (e.g. execution time, hardware counters, etc).
For that, it is necessary to actually observe running code through some form of dynamic analysis.
Various tools have been developed to capture this program information.
A majority of these tools are not able to combine both static and dynamic program information at the level of detail of our tool to understand characteristics resulting from the structure of the code.
These tools are often not used for application data collection on production systems for several varying reasons:
\begin{itemize}
\item They are not fully automated (i.e. transparent to the user)
\item Have high barriers to entry for users
\item Are not able to handle full production application code bases
\item Require significant user intervention (e.g. code restructuring, working with tools experts)
\item They are not available on all platforms
\end{itemize}

OpenAnalysis~\cite{Strout:2005}, Program Database Toolkit~\cite{Lindlan2000}, ROSE~\cite{Willcock:2009:RGP:1621607.1621611}, Hercules~\cite{kartsaklis2012hercules}, TSF~\cite{bodin1998user}, RTalk~\cite{SPE:SPE1035}, and CHiLL/Harmony~\cite{tiwari2009scalable} rely on compiler technology to gather program information and most of them are used for code transformations done by tools.
HPCToolkit's~\cite{Adhianto2010} \texttt{hpcstruct} component gathers some program traits from the binaries of applications by trying to reconstruct specific constructs like loop nests, however it requires reconstructing the programs after optimisations are performed which may not match the original source code or be able to detect the higher level features of languages due to information loss during lowering.
The Collective Tuning project~\cite{Fursin:2016} aims to create a database of program structure features and find compiler optimisations for performance, power, and code size.
The main goal is to collect program features for the purpose of feeding these back to the compiler optimiser, instead of being made understandable for human researcher consumption.
However, it was the efforts of cTuning's Interactive Compilation Interface~\cite{ctuning-ici} project that contributed to adoption of \acs{GCC}'s plugins. 

Dehydra~\cite{dehydra} and Treehydra~\cite{treehydra} are analysis plugins that expose different \acs{GCC} \acp{IR} intended for simple analyses and ``semantic grep'' applications.
Unfortunately, they have only limited Fortran 90 support, and the output hides important application information.
Pliny~\cite{Feser:2015} is a project that focuses on detecting and fixing errors in programs, as well as synthesising reliable code from high level specifications.
It relies on mining information and statistics and is both still in the early research stage and not meant for increasing program understanding.
It also currently doesn't support Fortran.

Finally, tools such as XALT/ALTD~\cite{xalt,xalt2}, PerfTrack~\cite{Karavanic:2005:IDT:1105760.1105804}, Oxbow/PADS~\cite{oxbowpads}, IPM~\cite{5695625}, and \acs{HPC} system scheduling information provide information on the system environment, linkage (e.g. for library detection), and runtime and performance that is complementary to application source code features.

Our focus on utilising the compiler for our static analysis means that we can benefit from the dearth of both static analysis as well as compiler research.
Compilers face many difficult challenges in transforming code while ensuring that the meaning of the code has not been changed.
What this means for us is that some of the same methods used to validate transformations can be applied to analysing code, such as for array accesses within loops.

\section{Source Code Analyser}
\label{sec:analysis}
For the automated analysis, rather than developing our own tools, we relied as much as possible on existing tools and environments.
%TODO -- you need to explain what information are you extracting from the compiler --- you need to mention that you are extracting the Abstract Syntax Tree, and what other analysis and that you are leverging from gfortran intermediate representation data structures, where you are making a bridge from the intermediate program representation to SQL.
Our general methodology was to extract data about the code directly from the compiler, insert it into tables in an \acs{SQL} database, separately run a dynamic analysis to collect periodic samples of where the application is spending time and also add it to the database, then finally run queries on the database combining the two sources of data to construct metrics on which to weigh the application's data structures.
These points will be addressed separately in the following subsections.

The desired result of this analysis is to achieve similar results to what the manual tweaking of our previous work found so that it can more realistically and scalably be applied to larger and more complex applications.
To this end, we will compare the results from our previous work to what we achieved with our new tools to determine how well they are able to match up.
We will then perform a reverse validation by applying the same analysis to a different and larger application that was not part of our previous work, modify the memory placement for the reported hotspots, and see how well the actual results fall in line with the expected results.
This will be described in more detail in Section~\ref{sec:results}.
\subsection{Static Analysis}
%TODO: particular spec/features of fortran?
For the static analysis, we relied on the \ac{GCC} for its support of Fortran.
%This seems to be not very interesting for the paper -- this is just mechanics (Loading a module)
Using its plugin \acs{API}, we created a module to be loaded and run after the compiler finishes parsing the code.
%TODO: poly
%Additionally, it also registers a pass to be run while it's generating its GIMPLE \ac{IR} in order to get additional analysis information from the lower stages of the compiler.
The plugin runs through all the relevant data structures representing the code to queue the data, then proceeds to dump them all to the database at once in bulk transactions.
For the database, we used PostgreSQL for some of its advanced query support.
\subsubsection{Dumping the Data}
We did not attempt to create our own schema to represent the code, as it would not be necessary due to the tables not being intended as a user-facing interface.
%TODO: no possessive acronyms... :'(
Instead, we simply dumped out \ac{GCC}'s internal data structures such that each data structure was a table, and each member of the data structure was a column.
%TODO: technically not all tables...mention enums?
Common to all tables are two columns for the pointer to the data structure in memory during compilation, and a build ID unique to each individual invocation of the compiler.
%TODO: necessary to very briefly explain primary/foreign keys, etc, or are they sufficiently self-describing?
Together, these two fields form the table's composite primary key to uniquely identify each record in the database.
Furthermore, this also allows us to dump all data in the most straightforward of manners - primitive types get dumped as their corresponding \acs{SQL} data type, and pointers to other objects as the raw value of those pointers.
\subsubsection{Querying the Database}
Due to the way we identify records and store object references, we are able to use any columns representing pointers as foreign keys with which to join connected tables in the database.
The first and most important thing we need the database to do for us is find and count all references to memory in the code.
Much of our focus on the static analysis is on the level of line numbers, so our queries were constructed so as to identify memory accesses and relate them to those line numbers.
In particular, we go through each statement and look up which file and lines they span using the line map table, traverse the expression trees for those statements, filter for expressions containing references to the symbol table, then finally check the symbol table to determine which particular symbols are being referenced.
In this way, we are then able construct queries to tally the total number of references for every symbol referenced across any subset of lines.

%TODO: more poly?
In order to enhance the predictive capacity of our analysis, we look to polyhedral analysis by utilising \acsp{GCC} Graphite framework\cite{trifunovic:inria-00551516}.
At this time, our use of the polytopes provided by Graphite is simply to do some basic reasoning about memory access order for arrays - notably, to find cases where a particular access is likely to result in either a greater or lesser number of cache misses than a typical in order access pattern.

We classify accesses into three categories based on a focus on the relation to the domain of the innermost loop.
For loops with iterators \texttt{i}, \texttt{j}, \texttt{k}, in that order, we refer to these categories as follows: the \texttt{[k][j]} class, the \texttt{[j][k]} class, and the \texttt{[j][i]} class.
In all cases, what we are really trying to address is the use of the innermost loop's iterator, or the lack thereof, so this is intended to serve as a simplification in that \texttt{j} is the "same" as \texttt{k + i} or \texttt{k - 1}.
Since Fortran indexes arrays in a column-major order, this means that the first \texttt{[k][j]} class is intended to represent accesses that are presumed to be both in order in memory and reasonably close in proximity between iterations, if not contiguous.
Similarly, the second \texttt{[j][k]} class represents the case where for sufficiently large data objects a new cache block may need to be fetched from memory for each iteration of the innermost loop, resulting in degraded performance compared to the first class.
Finally, the final \texttt{[j][i]} class is intended to represent the cases where the memory location being accessed remains constant throughout the entire innermost loop, only changing as often as any relevant enclosing loop iterates and thus carrying the expectation of being even slightly less expensive than the first class.
%TODO: reference this cost model paper in some way?
With these classifications, we can then weigh each access observed for a sample more or less based on the access type to create a cost model for our analysis.%\cite{5608348}
\subsection{Dynamic Analysis}
%TODO: cite map?
For the dynamic analysis, we used the ARM MAP sampling based profiler.
Given a sample frequency and number of samples, it periodically probes the application for a set of predetermined metrics, then gathers the samples together to dump them out to an \acs{XML} file.
What we are interested in here is simply where each thread is at in the code at the time of sampling, specifically the file and line number for the deepest frame of the stack.
We process the \acs{XML} file to extract this information via \ac{XSLT} to produce a \acs{CSV} file with the fields we want, and then upload it to the database so that we can achieve a count of how many times any given line is observed being executed by a thread.
We use this simple metric to determine in what regions of code the application is spending most of its time, so that we may focus only upon those regions.
\subsection{Combined Analysis}
The real value of the analysis comes when combining the previously described static and dynamic aspects into a more comprehensive evaluation of the code.
Using the results from MAP, we focus only on the subset of lines we recorded samples from, and use the information from the static analysis to obtain the symbols and their reference totals across those lines.
We treat all references as equal, with no regard as to whether they are read or write accesses or what level of the memory hierarchy their current/local value may be stored in at the time of access.
After we have all the reference count totals, we further sum all the totals and determine what percent of that total each symbol constitutes.
It is this percentage we use to determine the relative impact on memory performance in lieu of the benefit factor from our previous work as described in Section~\ref{sec:prevwork}.
%TODO: poly

\section{Case Study}
\label{sec:casestudy}
What follows is a series of queries for particular kinds of questions of interest for CP2K 
application users as well as within the \ac{OLCF}.
This case study is intended to show how the tool can be used to query an application and to 
demonstrate the level of detail of the queries that can be of general interest to any \acs{HPC} 
applications.

Queries such as these can be easily composed to work on different fields/attributes, combined, or 
broken apart in any number of ways to manipulate the content of the \ac{IR}, as well as made with or 
without the use of the dynamic sampling obtained from user profiling to narrow the 
search space based on the application hotspots.
It is worth noting that for added simplicity, we omit the parts of the queries that filter based upon 
build, application, or profiling IDs as these are unnecessary to understand the queries.

\subsection{CP2K}
\label{sec:cp2k}
CP2K \cite{hutter2014cp2k} is a popular molecular simulation package, written in Fortran 95, that includes numerous routines for quantum chemistry and solid-state physics, including \ac{AIMD} \cite{marx2009ab}.
Molecular dynamics within the \ac{BO} regime simulates atomic motion according to classical mechanics, while forces may be calculated using any type of physics.
For \ac{AIMD}, the forces are calculated using quantum mechanics, via derivatives of the system's electronic potential energy surface.
This energy is usually obtained with a \ac{SCF} calculation with \ac{DFT} \cite{vandevondele2012linear,hutter2014cp2k}, which involves an iterative solver that must converge before the energy is known, and tensor contractions of large tensors representing electron orbitals.
Therefore, each time-step requires considerably more computational time and memory than classical molecular dynamics.
However, \ac{AIMD} is able to explore essential chemical properties such as time-dependent changes in polarization, charge densities, protonation states, and bond breaking and formation, all of which are inaccessible to classical molecular dynamics and of great interest to physicists, chemists and molecular biologists.

Until recently, \ac{AIMD} simulations on systems with thousands of atoms for timescales longer than a few picoseconds was not possible.
Currently, several parallel implementations of the \ac{DFT} based \ac{SCF} calculation have exploited massive parallelization on \acs{HPC} systems to obtain the \ac{SCF} based electronic energy of such large systems in under a minute \cite{vasp_bench,kresse1996efficient,cp2k_bench,vandevondele2012linear}.
This opens up the possibility of simulating, for the first time, electronic-structure-dependent properties as time series extending to hundreds of picoseconds of simulation time, which can be compared to experimental results \cite{gillan2016perspective, pestana2017ab, hassanali2013proton, milovanovic2018new, sellner2013charge}.

The \ac{LS-SCF} calculation \cite{vandevondele2012linear} in CP2K has made use of an algorithm that exploits physical locality to decompose the \ac{SCF} calculation into what effectively becomes sparse matrix multiplication, which is then approached with their own auto-tuning small matrix multiplication library, \texttt{libsmm} \cite{borvstnik2014sparse}.
An Intel-optimized version of this library, \texttt{libxsmm}, has also been created for CP2K by programmers working for Intel \cite{heinecke2016libxsmm}.
In addition, a \acs{GPU} based version is available, and the procedures are all incorporated into the \ac{DBCSR} library \cite{borvstnik2014sparse,schutt2016gpu}, which is the engine for the \ac{LS-SCF} calculation.
Despite these efforts and other extensive optimizations including excellent \acs{MPI} based decomposition, OpenMP threading, and use of parallel libraries such as ScaLAPACK and threaded \ac{FFTW}, CP2K cannot scale to much more than a thousand nodes on supercomputers, resulting in a lack of ability to utilize over 5\% of systems such as the \acs{OLCF}'s Titan, while capability level programs often are able to use much more.
It is programs such as CP2K, therefore, which provide state of the art tools for obtaining cutting edge scientific results, that can benefit greatly from performance analysis tools beyond profilers, in order to potentially further increase their ability to make efficient use of leadership computing resources.


%TODO: NOTES: missing id hacks and omission of ids in listings, interprocedural analysis 
%discussion, library file/module discussion, ...?
\subsection{Finding OpenMP Regions of Interest}
Identifying important code regions to port to a new architecture is of extreme importance
for performance and to develop a porting plan for a new architecture.
This query incorporates static and some dynamic profiling 
data to determine where we might want to focus our analysis and/or porting efforts.
We address this problem in two separate steps - determining the characteristics with which we will 
select regions of interest, and finding the set of functions observed in our sampling data with which 
we will restrict our search.
The second part requires enough to merit its own query here, which can be seen in 
Listing~\ref{lst:restrict}.
In this query, we start by selecting all the sampled stack information from our profiler output, located 
in a table called \texttt{stack\_line\_frequency}.
We then proceed to ``unnest'' these stacks to get a series of individual file and line tuples, and match 
the resulting file names to those stored in the database so as to filter out any that weren't included in 
the original export of data (necessary since source level information is not necessarily available for 
all functions that may be in the stack, e.g. for external libraries).
Once we have this, we compare them against any code lines that contain matching line numbers and 
get the names for the containing functions.
Finally, we eliminate duplicates so that we have a distinct set of all functions sampled.

\begin{lstlisting}[caption=Determing Functions Sampled, label=lst:restrict]
select distinct code_lines.ns_proc_name.name
from	(
		select stack_files.file,
		       stack_lines.line
		from	(
				select   files,
				         lines
				from     map.stack_line_frequency
				group by files,
				         lines limit 1) stack
		join   unnest(stack.files) with ordinality as stack_files(file, index)
		on     true
		join   unnest(stack.lines) with ordinality as stack_lines(line, index)
		on     stack_lines.index = stack_files.index
		join	(
				select distinct filename
				from            gcc.gfc_file) file
		on     file.filename = stack_files.file) freq
join	gcc.code_lines as code_lines
on	code_lines.filename = freq.file
and	code_lines.lines @> array[freq.line]
join	gcc.gfc_symbol as sym
on	sym.record_address = code_lines.ns_proc_name
and	sym.build_id = code_lines.build_id;
\end{lstlisting}

Next, we incorporate the previous query into our search as a subquery, which we will simply refer to 
as \texttt{subquery} for simplicity.
This query can be seen in Listing~\ref{lst:omp-search}.
We decided to search among the functions for those that had high numbers of OpenMP parallel regions, under 
the logic that they might have the most complex or interesting behaviour.
To do this, we search the \texttt{annotated\_code} table for statements that define OpenMP regions 
(i.e., have a reference to a \texttt{gfc\_omp\_clauses} entry in their \texttt{ext.omp\_clauses} field) 
that have an operation type of 75 or 76 (EXEC\_OMP\_PARALLEL and 
EXEC\_OMP\_PARALLEL\_DO, respectively).
Finally, we count the number of such regions for each function and filter for only those that both are 
included in the sampled data and have more than one parallel region, sorting by the number of 
regions counted.
The results of this query can be seen in Table~\ref{tbl:parallel-regions}.

\begin{lstlisting}[caption=Finding High Concentrations of OpenMP Parallel Regions, 
label=lst:omp-search]
select 
sym.name,
       count(sym.name)
from   gcc.annotated_code as annotated
       join gcc.gfc_symbol as sym
         on sym.record_address = annotated.ns_proc_name
            and sym.build_id = annotated.build_id
where  annotated."ext.omp_clauses" is not null
       and annotated.op in ( 75, 76 )
       and sym.name in (select *
                        from   subquery)
group  by sym.name
having count(sym.name) > 1
order  by count(sym.name) desc;  
\end{lstlisting}

\begin{table}[htbp]
\caption{Subroutines with High Numbers of OpenMP Parallel Regions}
\begin{center}
\begin{tabular}{|c|c|c|}
\hline
\textbf{Subroutine} & \textbf{Parallel Regions} \\
\hline
pw\_axpy & 14 \\
\hline
xc\_vxc\_pw\_create & 10 \\
\hline
pw\_copy & 8 \\
\hline
pw\_scatter\_s & 4 \\
\hline
rs\_pw\_transfer & 4 \\
\hline
dbcsr\_data\_copyall & 4 \\
\hline
multiply\_cannon & 4 \\
\hline
pw\_gather\_s & 2 \\
\hline
dbcsr\_multiply\_generic & 2 \\
\hline
build\_core\_ppnl & 2 \\
\hline
\end{tabular}
\label{tbl:parallel-regions}
\end{center}
\end{table}

\subsection{Measuring Variable Use Within OpenMP Parallel Regions}

Finally, we will combine elements of the previous queries to accomplish something truly novel - 
getting a rough measure of performance for each of hundreds of individual variables in OpenMP 
regions based on standard profiler sampling and measuring the effects of varying numbers of 
threads on these measures.
We accomplish this by using the line-based reference counting described in 
Section~\ref{sec:querying}, parts of the previous query to both join the static and dynamic data and 
filter for references occurring within OpenMP regions, as well as the filtering based on function name 
that we first used for finding ScaLAPACK function calls in Table~\ref{tbl:scalapack-funcs}.
Thus, we:
\begin{itemize}
\item Use the output from MAP to get all the sampled stacks from the application and how many 
times they were encountered
\item Match up the file and line numbers to the database to determine which functions those samples 
occurred in
\item Select only the samples that were contained within the top few functions of interest (where we 
observed from profiling that the vast majority of time was spent)
\item Find all variables referenced on those lines and add the sample count from MAP to a running 
total for each variable
\item Group the data together by variable (not lines) and add the counts together for every time each 
variable was accessed across all relevant lines
\item Repeat for the data obtained from multiple runs with different thread counts (1, 2, 4, and 8)
\end{itemize}
The results of this query can be seen in Figure~\ref{fig:openmp-refcount}, where each line 
represents a single variable and the total number of times any thread accessed it across the 
observed samples, adjusted proportionately for thread count.
If all parallel regions had a fairly even work distribution across threads and we achieved linear 
speedup, we might expect to see the corresponding lines in the plot be largely flat, representing 
roughly the same total number of accesses but with greater numbers of threads making those 
accesses concurrently.
This also means that what we don't want to see is deviations resulting in increases to the number of 
accesses.
In the plot, we can see that this is the case for a fair number of variables in moving from one to two 
threads, but that access counts for most variables remain fairly constant afterward.
This may be attributable to startup overhead, though the affected variables merit closer investigation.

\begin{figure}
\begin{center}
\includegraphics[width=0.45\textwidth]{images/cp2k-omp-inc-full.pdf}
\end{center}
\caption{Sampled Reference Counts to Variables Within OpenMP Parallel Regions}
\label{fig:openmp-refcount}
\end{figure}

\subsection{Determining Use of System and User Libraries}
One of the bigger concerns for \acs{HPC} applications when targeting a new system is determining 
which system resources, 
particularly that of libraries, are actively being used in applications so as to better direct porting 
efforts.
As such, it is valuable to query not just which libraries are linked in (which has already been done by  
tools like XALT), but also the context in which functions within those libraries are actually getting 
called within user code and their call paths.
To this end, we queried CP2K to discover which \ac{FFTW} functions were actually referenced in the 
code.
The query can be seen in Listing~\ref{lst:fftw}, which searches for call statements and joins each one 
with the symbol table on its \texttt{resolved\_sym} field, which is 
used for referencing the called function's symbol, before finally outputting a sorted list of all distinct 
names found for symbols originating from the library module \texttt{fftw3\_lib}.
The results can be seen in Table~\ref{tab:fftw-funcs}.

\begin{lstlisting}[caption=Querying for Use of Library Functions, label=lst:fftw]
select distinct sym.name
from   gcc.gfc_code as code
       join gcc.gfc_symbol as sym
         on sym.record_address = code.resolved_sym
            and sym.build_id = code.build_id
where  code.op = 10
       and sym.module = 'fftw3_lib'
order  by sym.name;
\end{lstlisting}

\begin{table}[htbp]
\caption{FFTW Subroutines Referenced}
\begin{center}
\begin{tabular}{|c|}
\hline
\textbf{Subroutine} \\
\hline
fftw31dm \\
\hline
fftw33d \\
\hline
fftw3\_compute\_rows\_per\_th \\
\hline
fftw3\_create\_3d\_plans \\
\hline
fftw3\_create\_guru\_plan \\
\hline
fftw3\_create\_plan\_1dm \\
\hline
fftw3\_create\_plan\_3d \\
\hline
fftw3\_destroy\_plan \\
\hline
fftw3\_do\_cleanup \\
\hline
fftw3\_do\_init \\
\hline
fftw3\_get\_lengths \\
\hline
fftw3\_workshare\_execute\_dft \\
\hline
fftw\_cleanup \\
\hline
fftw\_export\_wisdom\_to\_file \\
\hline
fftw\_import\_wisdom\_from\_file \\
\hline
sortint \\
\hline
\end{tabular}
\label{tab:fftw-funcs}
\end{center}
\end{table}

The same general form of query could be narrowed down in a couple other ways as well.
In addition to filtering function symbols by module(s), one could also filter based on which file(s) they 
are found in.
A user may also want to know more detailed information about invocations, particularly the file/line 
locations each can be found at.
Table~\ref{tbl:scalapack-funcs} shows an example of exactly this, showing the file and line numbers 
found for calls to ScaLAPACK subroutines.
The results have been limited to the top 15 for brevity.

\begin{table}[htbp]
\caption{ScaLAPACK Subroutine Reference Locations}
\begin{center}
\begin{tabular}{|c|c|c|}
\hline
\textbf{File} & \textbf{Lines} & \textbf{Subroutine} \\
\hline
src/cp\_dbcsr\_cholesky.F & {103} & pspotrf \\
\hline
src/cp\_dbcsr\_cholesky.F & {105} & pdpotrf \\
\hline
src/cp\_dbcsr\_cholesky.F & {187} & pspotri \\
\hline
src/cp\_dbcsr\_cholesky.F & {189} & pdpotri \\
\hline
src/fm/cp\_cfm\_basic\_linalg.F & {411} & pzgetrf \\
\hline
src/fm/cp\_cfm\_basic\_linalg.F & {703} & pzgetrf \\
\hline
src/fm/cp\_cfm\_basic\_linalg.F & {736,737} & pzgetrs \\
\hline
src/fm/cp\_cfm\_basic\_linalg.F & {799,800} & pzgetrf \\
\hline
src/fm/cp\_cfm\_basic\_linalg.F & {812,813} & pzgetri \\
\hline
src/fm/cp\_cfm\_basic\_linalg.F & {818,819} & pzgetri \\
\hline
src/fm/cp\_cfm\_basic\_linalg.F & {882} & pzpotrf \\
\hline
src/fm/cp\_cfm\_basic\_linalg.F & {938} & pzpotri \\
\hline
src/fm/cp\_cfm\_basic\_linalg.F & {1147} & pztrtri \\
\hline
src/fm/cp\_cfm\_diag.F & {90,91} & pzheevd \\
\hline
src/fm/cp\_cfm\_diag.F & {121,122} & pzheevd \\
\hline
%src/fm/cp\_fm\_basic\_linalg.F & {300} & pdgetrf \\
%\hline
%src/fm/cp\_fm\_basic\_linalg.F & {1348} & pdgetrf \\
%\hline
%src/fm/cp\_fm\_basic\_linalg.F & {1383} & pdgetrs \\
%\hline
%src/fm/cp\_fm\_basic\_linalg.F & {1400,1401} & pdgesvd \\
%\hline
%src/fm/cp\_fm\_basic\_linalg.F & {1406,1407} & pdgesvd \\
%\hline
%src/fm/cp\_fm\_basic\_linalg.F & {1511} & pdtrtri \\
%\hline
%src/fm/cp\_fm\_basic\_linalg.F & {1576} & pdgeqrf \\
%\hline
%src/fm/cp\_fm\_basic\_linalg.F & {1580} & pdgeqrf \\
%\hline
%src/fm/cp\_fm\_basic\_linalg.F & {1640} & pdgetrf \\
%\hline
%src/fm/cp\_fm\_basic\_linalg.F & {1641,1642} & pdgetrs \\
%\hline
%src/fm/cp\_fm\_basic\_linalg.F & {1776,1777} & psgetrf \\
%\hline
%src/fm/cp\_fm\_basic\_linalg.F & {1779,1780} & pdgetrf \\
%\hline
%src/fm/cp\_fm\_basic\_linalg.F & {1804,1805} & psgetri \\
%\hline
%src/fm/cp\_fm\_basic\_linalg.F & {1810,1811} & pdgetri \\
%\hline
%src/fm/cp\_fm\_basic\_linalg.F & {1820,1821} & psgetri \\
%\hline
%src/fm/cp\_fm\_basic\_linalg.F & {1823,1824} & pdgetri \\
%\hline
%src/fm/cp\_fm\_cholesky.F & {75} & pspotrf \\
%\hline
%src/fm/cp\_fm\_cholesky.F & {77} & pdpotrf \\
%\hline
%src/fm/cp\_fm\_cholesky.F & {143} & pspotri \\
%\hline
%src/fm/cp\_fm\_cholesky.F & {145} & pdpotri \\
%\hline
%src/fm/cp\_fm\_cholesky.F & {211} & pdsygst \\
%\hline
%src/fm/cp\_fm\_diag.F & {359,360} & pdsyevd \\
%\hline
%src/fm/cp\_fm\_diag.F & {380,381} & pdsyevd \\
%\hline
%src/fm/cp\_fm\_diag.F & {552,553} & pdsyevx \\
%\hline
\end{tabular}
\label{tbl:scalapack-funcs}
\end{center}
\end{table}

\subsection{Determining Performance Impact of Libraries}

We have seen that we can check for library subroutine references and determine which ones are actually used as well as where, but what can be even more interesting is determining the respective impact of the libraries on overall execution performance.
Pairing the information on the stack from tracing with link-time data from XALT, it is possible for us to break down how much of the runtime is spent on code within specific libraries or other component archives of the application.
In essence, we can pair observed performance metrics with any statically or dynamically linked objects responsible for them in order to gain a better understanding of which libraries/modules have the greatest influence on execution, and direct focus there.

In order to do this, we first must determine in which source file a sample's stack terminates in from the profiling data.
We can then query the database to find the build IDs generated for those samples and use our previously described integration with XALT to find the corresponding objects they got linked into.
If we then use these objects as the basis for processing our sample data, we can see which objects dominate performance the most.
Results of applying this process to CP2K can be seen in Figure~\ref{fig:library-performance}.

This analysis can be particularly valuable for applications like CP2K due to the use and static linking of multiple different projects as libraries internal to the build, like the \ac{DBCSR} library.
These would not ordinarily be easily distinguishable from each other or other parts of the main application with most alternative analysis methods, making it more difficult to determine what components are responsible for different observed performance characteristics, especially as the number of moving parts increases.

\begin{figure}
\begin{center}
\includegraphics[width=0.45\textwidth]{images/library-performance.png}
\end{center}
\caption{Percent of Samples by Library}
\label{fig:library-performance}
\end{figure}

\section{Conclusion and Future Work}
\label{sec:conclusion}
With this tool, we are able to gain new insight into applications and explore code and performance data in more ways and with a far finer level of detail than can be achieved with traditional analysis tools.
Users can benefit by using it to explore and analyse their code in a way that may have not been possible before, combining profiling data with structural data on the code to investigate things like memory and data structure usage patterns.

While there is already a very large degree of information this tool can provide and an unprecedented degree of flexibility, there are a still lot of directions this work could go, and a number of things necessary to help it get there.
One of the more notable such tasks is to actually reduce the level of information extracted down to more closely match what is actually utilised by end users as well as create a better and more language agnostic schema designed specifically with the tool in mind.
Both of these points result from the fact that the tool currently extracts internal C data structures within \acs{GCC} as is, which results in both a number of fields and storage methods for those fields as well as layout across data structures that may not always make quite as much sense for an \acs{SQL} database as it may for C code.
Additionally, once a mostly or entirely language agnostic schema is created, it can also pave the way toward easily dropping in support for C/C++ applications as well.
However, the best way to determine which fields are important and create such a schema, more and in-depth real world analysis of code is necessary in order to gain a better idea of what works and is needed.

Another future direction is to investigate polyhedral analysis within the scope of this tool, and determine if and how it could be usefully applied to queries to enhance analysis.
Additionally, most of the queries we have constructed thus far have still remained confined within the scope of functions, so an alternative direction may be to explore more of an interprocedural analysis route.
Though this is well known for being a difficult problem to tackle, the level of detail and structure within this tool may provide some unique and unprecedented ways to perform such queries across function boundaries.

A more ambitious possibility is to integrate data mining/machine learning techniques into the tool's flow to look for common patterns.
This could potentially create the capability to not only use these methods to further help direct focus toward predicted points of interest in a more automated fashion, but could conceivably open the doors to finding large scale patterns across many separate applications and their different iterations and runtime environments.
If successful, it could have massive implications for \acl{HPC} and its associated science.

\section*{Acknowledgements}
\label{sec:ack}
This work is supported by the United States \ac{DoE} and used resources of the \aclp{CRDP} and the \ac{OLCF} at \acl{ORNL}.

\bibliographystyle{IEEEtran}
\bibliography{references}
\end{document}
