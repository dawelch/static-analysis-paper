%We need to change the text to reflect
Emerging HPC platforms are becoming extraordinarily difficult to program as a result of systems with diverse architectures, featuring different types of memories on the node, the presence of multi-core and accelerator architectures, system designs with fewer nodes each with multiple accelerators (i.e. fat nodes), and those with large numbers of nodes consisting of multi-core \acs{CPU} (i.e. thin nodes).
New programming models and libraries are being developed to address performance portability challenges and aid in porting efforts of application developers, but substantial code restructuring is still necessary to fully make use of these new technologies in an performance portable way.
To do this effectively, developers need information about their source code characteristics including static and dynamic (e.g. performance) information to direct their optimisation efforts and make key decisions.
In this paper we describe a tool that combines compiler and profiler information to query performance characteristics within an application or across multiple applications on a given platform.
Static and dynamic data about applications is collected and stored together in an \acs{SQL} database that can be later queried to study application characteristics and patterns.
We will demonstrate the capabilities of this tool with an application-driven case study that aims at understanding application code and its use of scientific libraries via a real world example from the molecular simulation application CP2K.
