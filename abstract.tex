Emerging HPC platforms are becoming extraordinarily difficult to program as a result of complex, deep and heterogeneous memory hierarchies, heterogeneous cores, and the need to divide work and data among them.
New programming models and libraries are being developed to aid in porting efforts of application developers, but substantial code restructuring is still necessary to fully make use of these new technologies.
To do this effectively, these developers need information about their source code characteristics, including static and dynamic (e.g. performance) information to direct their efforts and make key decisions.
On the other side, system administrators also need to understand how users are using the software stack and resources to exploit modern architectures and how their investment improves the productivity of the users on a platform.

In this paper we describe a tool that we are constructing to query program information at the level of an application or across multiple applications.
Static and dynamic data about applications are collected and stored together in an \acs{SQL} database that can be queried by either a developer or a system administrator.
We will demonstrate the capabilities of this tool via a real-world example from application-driven case studies that aims at understanding the use of scientific libraries on a routine from the molecular simulation application CP2K.
