%We need to change the text to reflect
Emerging HPC platforms are becoming extraordinarily difficult to program as a result of systems with diverse node architectures (e.g. with different types of memories, accelerators, \acs{CPU} multi-cores, etc), some with a small number of ``fat" heterogenous nodes (with multiple accelerators) and others with a large number of ``thin" homogenous nodes consisting of multi-core \acs{CPU}s.
New programming models are being developed to address performance portability challenges as well as the tuning and port of scientific libraries that application can use to exploit  these architectures efficiently. 
To do this, developers need information about their source code characteristics including static and dynamic (e.g. performance) information to refactor the code, understand their libraries usage as well as program information to direct their optimization efforts and make key decisions.
In this paper we describe a tool that combines compiler and profiler information to query performance characteristics across multiple platforms.
Static and dynamic data about applications is collected and stored together in an \acs{SQL} database that can be later queried to study application characteristics and patterns.
We will demonstrate the capabilities of this tool with an application-driven case study that aims at understanding application code and its use of scientific libraries via a real world example from the molecular simulation application CP2K.
