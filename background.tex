% 2 pages
Describe the challenges posed by new platforms in more detail, describe what programming models and libraries are doing to address this - which includes stating that the user will need to remain involved, think of OpenMP memory features - and then what existing tools offer
Hi Oscar,

Yes, I actually don't know if the linear scaling SCF calculation even uses scalapack at all. But building cp2k in parallel requires scalapack; however, that might be for other routines. 

I'm very interested in how the use of scalapack and blas in general affects this particular calculation. Understanding the parallel-communication layers involved with blas/lapack calls would be great too. Also, the use of the fftw library.

The Eos build outperforms the Titan build, even with the GPU acceleration on Titan. The CPU version is really high performance for this type of code. It's a poster child for a great code that needs some help to be able to run better on Summit and future systems. It was optimized on Piz Daint, so a lot of reliance on intel-based programming models and MKL. I would like to be able to swap these out and still have equal performance on other systems.

I would like to know why this is; my guess is that they have made heavy use of intel intrinsics, and/or the MKL library is adding to efficiency: I'm wondering if presence of MKL blacs/scalapack and MKL fftw threads in my Eos build may be giving it that extra boost. 

If this is so, then things can possibly be done (at some point) to create/incorporate similar libraries, and maybe even GPU-based libraries on Summit, as we discussed, for instance with a GPU-based blacs/scalapack-type library optimized for the Summit node architecture. Currently essl does not have its own hardware-optimized scalapack or blacs, and this may be a problem for codes like this. Furthermore, the need for vendor-provided scientific libraries at the scalapack/blacs level seems to have been omitted from the requirements for IBM. So giving this trace to the people responsible for communicating our needs to IBM may be helpful.

The dbcsr library, which is called the most frequently in this calculation, most probably uses blas of some kind, I can't imagine they coded all the linear algebra from scratch. So tracing the scientific library calls made by dbcsr would be a very important thing to do. Improving cp2k performance for this LS-SCF calculation relies on improving the dbcsr.

One last thing that I wouldn't mind knowing about, if it's possible, is how the code handles openMP vs. MPI use. My benchmarks of this calculation on Eos and Titan showed that using many MPI ranks per node, even 32 on Eos, gave better performance than any use of openMP at all! I thought this was interesting. It's possible that the MPI implementation is really well done and the openMP is not. Or, it could just be something about the calculation itself. I think this is actually a very interesting CS problem.

Thanks!

-A
