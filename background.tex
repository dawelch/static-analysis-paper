%TODO: themes - portability, scalability
%TODO: not sure if these are the best citations here, and/or need more static analysis references, perhaps for a couple other non-compiler forms of static analysis? language/design related?
%TODO: This looks starting from sequential code
%TODO: Challenge is calculating system dependences: control flow, data flow, loop dependences -- all with alias analysis and array section analysis.
%TODO: Its hard today as there are limitations for these types of analysis: inter-procedural information -- how to propagate information across procedures through arguments or globals, etc.
%TODO: This gets even harder for parallel analysis where there are memory syncrhonizations, SPMD executions (threads taking different branches, etc).
%TODO: Historically, compilers optimize better loops or apply loop transformations rather than data structure optimizations because
%TODO: they tend to be more localized. 
%TODO: Early successes has been using sequential analysis to parallelize via automatic vectorization but this only focus on inner loops
%TODO: This has demostrated the importance of good dependence analysis and also good cost-models to measure the profitability of SIMD.
%TODO: To improve this, we need to look at optimizing using runtime information. This includes feedback directed optimizations (online and offload), where runtime functional data of the program is calcuated (frequency, value or phased behavior) to detect
%TODO: Hotspots of the code. Little of these feedback optimizaitons has focus on memory accesses. 

%Summarize all this paragraph in two sentences
Research on static analysis of code has a very long history and has both taken many forms and addressed innumerable problem domains\cite{Andrade:2012:SAW:2355585.2355593}\cite{1194988}.
Our focus on utilising the compiler for our static analysis means that we can benefit from the dearth of both static analysis as well as compiler research.
Compilers face many difficult challenges in transforming code while ensuring that the meaning of the code has not been changed.
%TODO: did I actually mention focusing on arrays/loops in the intro?  I don't think I did...
What this means for us is that some of the same methods used to validate transformations can be applied to analysing memory use, such as for array accesses within loops.
A key concept to that end is that of dependences between statements, such that the order of statements required to preserve this meaning must remain the same.
An integral process used to determine these dependences is to determine how variables are used and data travels through them in some form of dataflow analysis\cite{Feautrier1991}.
This is also commonly used to discover more complicated patterns of memory access, for enabling optimisations such as loop tiling to make better use of cache performance.
%TODO: mention other things, such as direction/distance vectors?

Among the more advanced methods compilers use to analyse code is to employ the polyhedral model\cite{Cousot:1978:ADL:512760.512770}\cite{Bagnara:2009:APC:1628316.1628385}\cite{benabderrahmane.10.cc}.
The polyhedral method is ideally suited for representing and reasoning about loops, although is generally restricted to operating on affine loop nests.
A primary benefit of using polytopes over other methods is their natural ability to compactly and mathematically represent access patterns within loops regardless of the bounds of their domains.
The model constructs polytopes for the $n$-dimensional space reflected by the loop nest's domain where iterations are represented as lattice points.
As a polytope is composed of the solutions to a finite number of linear inequalities, much of the analysis becomes a series of math problems on matrices.
While the model is highly expressive and powerful, its applicability constraints and relative computational expense for some operations on it have traditionally limited its practical use in compilation\cite{DBLP:journals/entcs/Simon10a}.
Nonetheless, a lot of work has been done on it for more precisely determining dependences\cite{Vasilache:2006:VDA:1183401.1183448} as well as for more advanced optimisation techniques\cite{Nieuwenhuizen2014AutovectorizationUP}\cite{5260526}.
While not heavily employed, progress into polyhedral analysis has made its way into both the \ac{GCC}\cite{trifunovic:inria-00551516} and LLVM\cite{grosser.11.impact} compilers.
Here, the focus is on a subset of loop nests referred to as \acp{SCoP}\cite{TBas}, which are defined in \cite{benabderrahmane.10.cc} as the maximal set of consecutive statements where loop bounds and conditionals are affine functions of the surrounding loop iterators and parametres.
%TODO: reference openscop? openscop\cite{Bas11}
%TODO: talk more about gcc internals/analysis?  gdfa citation?

Another variety of techniques are that of \acp{FDO}, which aim to improve the execution behaviour of code based on information on its current or previous runtime behaviour\cite{Smith:2000:OCF:351403.351408}.
Runtime information, which may be specific to a given instance of an application's execution, helps the compiler direct its efforts to frequently executed regions of code and make better judgements on what set of optimisations can improve the code.
There is a large body of work on \acp{FDO}, including our own\cite{10.1007/978-3-540-68555-5_22} that focuses on offline optimisations.
Typical optimisations include feedback directed inlining, partial dead code elimination, instruction scheduling, code reordering and loop optimisations.
Several sets of runtime information based on training sets of input data may be used to characterise the typical runtime behaviour of the application.
A disadvantage of offline feedback optimisation is that it recompiles the application based on the training set.
Some of these systems rely on profiling in which the data is aggregated across multiple runs making the optimisation less specialised.
%A plus point: can do a lot of performance gathering

However, \acp{FDO} are not limited to post mortem optimisations.
%, such is the case of applications that adapts at runtime via parameritations or multiversioning, or partially compiled applications need to generate or reoptimize code on the fly.
Online feedback optimisations are attractive because they are less time consuming for the user, and may lead to specialised versions of code for the specific execution context.
%Didn't understand this point: The work of user-initiated recompilation, data gathering, source code modification, is leveraged.
There has been extensive research on using the runtime environment to dynamically guide the compilation and execution of code.
Much of this work has been carried out in the context of virtual machine technologies, typically those supporting the Java programming language\cite{Alpern:2000:JVM:1011388.1011400} and C.
In general, the runtime environment receives statically compiled code which may be in a native representation, or, as with the Java virtual machine, it may be in an intermediate format (e.g. Java bytecode) for portability and/or to retain desired semantic information.
In the latter case, it must be translated to the native code either using an interpreter or (more commonly) a \ac{JIT} compiler.
In some implementations, some portions of the code are statically predicted to rarely execute; these may remain uncompiled to save time and space.
A mechanism would then have to be in place to generate this code at runtime if necessary.

A common strategy for online \ac{FDO} is to selectively recompile identified ``hot spots'' with more aggressive optimisations.
Because applications generally spend the majority of their execution time in a relatively small fraction of the code, the typical selection procedure is to determine and target the available optimisations toward these ``hot spots'' in the running code.
Developing techniques for selecting hot spots and performing the optimisations in an efficient manner is an important research area.

%TODO: talk about vectorisation?  or...?

%polyhedral predictive modeling\cite{park.13.ijpp}
%trace-based vectorisation potential\cite{holewinski.12.pldi}
%machine learning for auto-vectorisation\cite{stock.12.taco}

%TODO: what level of detail to go into dynamic analysis?  I don't really have as much on that side and am at a bit of a loss for what should be discussed here

%array dataflow analysis in x10\cite{Yuki:2013:ADA:2517327.2442520}
%TODO: use this? probabilistic miss equations\cite{1183947}

%TODO: static memory access pattern analysis on a massively parallel gpu
%TODO: hierarchical memory systems/sparse linear systems\cite{Wang:1999:PEM:1080585.1080587}
