% 1/2 page - Oscar
Various tools have been developed to capture program information, but they are not commonly used for application data collection on production systems because they are either not fully automated (e.g. transparent to the user), have high barriers to entry for users, not able to handle full production application code bases, require significant user intervention (e.g. code restructuring, working with tools experts), and/or they are  not available on all platforms. Also, many of them are not able to combine both static and dynamic program information at the level of detail of our tool. OpenAnalysis~\cite{Strout:2005}, Program Database, Toolkit~\cite{Lindlan2000}, ROSE~\cite{Willcock:2009:RGP:1621607.1621611}, Hercules~\cite{kartsaklis2012hercules}, TSF~\cite{bodin1998user} and RTalk~\cite{SPE:SPE1035}, CHiLL/Harmony~\cite{tiwari2009scalable} rely on compiler technology to gather program information and most of them are used for code transformations done by tools. HPCToolkit~\cite{Adhianto2010}  
hpcstruct component gathers some program traits from the binaries of applications by trying to reconstruct specific constructs like loop nests, however it
it requires reconstructing the programs after optimizations are performed which may not match the original source code or 
cannot detect the higher level features of languages due to information loss during lowering. The Collective Tuning project~\cite{Fursin:2016} aims to create a database of program structure features and find compiler optimizations for performance, power, and code size. The main goal is to collect program features for the purpose of feeding these back to the compiler optimizer, instead of being made understandable for human researcher consumption. However, it was the efforts of cTuning's Interactive Compilation Interface~\cite{ctuning-ici} project that contributed to adoption of GCC's plugins. 

Dehydra~\cite{dehydra} and Treehydra~\cite{treehydra} are analysis plugins that expose different GCC intermediate representations intended for simple analyses and “semantic grep” applications. Unfortunately, they have only limited Fortran90 support, and the output hides important application information. Pliny~\cite{Feser:2015} is a project that focuses on detecting and fixing errors in programs, as well as synthesizing reliable code from high-level specifications. It relies on mining information and statistical information and is still in the early research stage and is not meant for program understanding. It also currently doesn't support Fortran. Finally, tools such as XALT/ALTD~\cite{xalt,xalt2}, PerfTrack~\cite{Karavanic:2005:IDT:1105760.1105804}, Oxbow/PADS~\cite{oxbowpads}, IPM~\cite{5695625}, and HPC system scheduling information provide system environment, linkage information (e.g. for library detection) runtime and performance information that is complementary to application source code features. 
